\documentclass{article}
\usepackage{graphicx}
\usepackage[utf8]{inputenc}
\usepackage{listings}
\usepackage{color}
\usepackage{xcolor}
\usepackage{textcomp}

\definecolor{solarized@base03}{HTML}{002B36}
\definecolor{solarized@base02}{HTML}{073642}
\definecolor{solarized@base01}{HTML}{586e75}
\definecolor{solarized@base00}{HTML}{657b83}
\definecolor{solarized@base0}{HTML}{839496}
\definecolor{solarized@base1}{HTML}{93a1a1}
\definecolor{solarized@base2}{HTML}{EEE8D5}
\definecolor{solarized@base3}{HTML}{FDF6E3}
\definecolor{solarized@yellow}{HTML}{B58900}
\definecolor{solarized@orange}{HTML}{CB4B16}
\definecolor{solarized@red}{HTML}{DC322F}
\definecolor{solarized@magenta}{HTML}{D33682}
\definecolor{solarized@violet}{HTML}{6C71C4}
\definecolor{solarized@blue}{HTML}{268BD2}
\definecolor{solarized@cyan}{HTML}{2AA198}
\definecolor{solarized@green}{HTML}{859900}

\lstset{
  language=Python,
  upquote=true,
  columns=fixed,
  tabsize=2,
  extendedchars=true,
  breaklines=true,
  frame=single,
  numbers=left,
  numbersep=5pt,
  rulesepcolor=\color{solarized@base03},
  numberstyle=\tiny\color{solarized@base01},
  basicstyle=\footnotesize\ttfamily,
  keywordstyle=\color{solarized@green},
  stringstyle=\color{solarized@cyan}\ttfamily,
  identifierstyle=\color{solarized@blue},
  commentstyle=\color{solarized@base01},
  emphstyle=\color{solarized@red}
}
\begin{document}

\title{Tarea Neville}
\author{Angel Caceres Licona}

\maketitle


\section{Haga un programa que obtenga las aproximaciones recursivas...}

\begin{lstlisting}
from sympy import symbols, init_printing, lambdify, horner, expand, pprint
import numpy as np
from numpy import zeros, diag

def polneville(x, y, x0):
    Q = np.zeros((len(x), len(x)), dtype=float)
    for k in range(0, len(x)):
        Q[k][0] = y[k]

    for i in range(1, len(x)):
        for j in range(1, i + 1):
            Q[i][j] = ((x0-x[i-j])*Q[i][j-1]-(x0-x[i])*Q[i-1][j-1])/(x[i]-x[i-j])

    return [Q[-1][-1], np.diag(Q, 0)]

datosx = np.array([8.1, 8.3, 8.6, 8.7], dtype=float)
datosy = np.array([16.94410, 17.56492, 18.50515, 18.82091], dtype=float)

interpolar = 8.4
valorN = polneville(datosx, datosy, interpolar)[0]
pprint('Para {0} obtenemos {1} con Neville. '.format(interpolar, valorN))
\end{lstlisting}

\section{Use el método de Neville para obtener aproximaciones...}
a)\\
Para el polinomio de grado 1 obtuve: $f(8.4) = 17.878330000000002$\\
Para el polinomio de grado 2 obtuve: $f(8.4) = 17.87713$\\
Para el polinomio de grado 3 obtuve: $f(8.4) = 17.877142500000005$\\

b)\\
Para el polinomio de grado 1 obtuve: $f(-\frac{1}{3}) = 0.014079166666666672$\\
Para el polinomio de grado 2 obtuve: $f(-\frac{1}{3}) = 0.01283680555555556$\\
Para el polinomio de grado 3 obtuve: $f(-\frac{1}{3}) = -0.04877314814814815$\\

\section{Use el método de Neville para aproximar $\sqrt{3}$...}

Para el polinomio de grado 4 obtuve: $f(\frac{1}{2}) = 1.7083338515625$\\

\section{Use el método de Neville para aproximar $\sqrt{3}$ usando $f(x)=\sqrt{x}$...}

Para el polinomio de grado 4 obtuve: $f(3) = 1.6906067646231164$\\

\end{document}